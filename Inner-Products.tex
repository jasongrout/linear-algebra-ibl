\chapter{Inner Products}

This is its own chapter, but could also be sprinkled in among the
other sections.  Advantages for dispersing this chapter is that it
slows down the course (so more time for retention) and it provides
some idea of naturally linearly independent vectors.  Of course, it
also distracts from the core focus on linear combinations and
transformations earlier, so that is an advantage for putting it all
here.

\section{Outline}
\begin{enumerate}
  \item Calculate inner product, length, angle, orthogonality, and
    projection in each space (except formal power series).  Examine
    these geometrically in $\RR^n$.  This introduces trace and
    transpose for the matrix space.
\item Examples where the inner product is not the dot product on $\RR^n$
\item Orthogonal complement; projecting onto a subspace
\item Least Squares
  \item When you have orthonormal bases, finding coordinates becomes
    much easier
  \item Making a basis orthonormal
  \item If the basis is orthonormal, the inverse transformation (matrix)
    is $A^T$.
\item Symmetric matrices and spectral theorem, diagonalization

\end{enumerate}

%%% Local Variables: 
%%% mode: latex
%%% TeX-master: "linear-algebra"
%%% End: 
