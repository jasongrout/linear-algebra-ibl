\chapter{Linear Combinations}

This chapter is short.  Very short.  One nice thing is that you can
drill home the concept of linear combinations and have a quiz before
moving on.  Or you can introduce the concept of inner product.

\section{Outline}
  Do each of these in $\RR^n$, $P_n$, $\RR^{m\times n}$, and formal
  power series rings.  Maybe also for $4\times 4$ pixel images
  (represented as matrices)?
  \begin{enumerate}
  \item Define linear combination.
  \item See geometry in $\RR^n$
  \end{enumerate}

\section{Linear Combinations}

\begin{definition}\note{More weird examples of vector spaces on
    \href{http://math.stackexchange.com/questions/5233/vivid-examples-of-vector-spaces}{MathOverflow}
    (and
    \href{http://math.stackexchange.com/questions/37871/non-numerical-vector-space-examples}{here}
    and \href{http://math.stackexchange.com/questions/4694/what-are-some-alternative-definitions-of-vector-addition-and-scalar-multiplicati}{here}.}
  The word ``vector'' means different things in different contexts:
  \begin{enumerate}
  \item A vector in $\RR^n$ is a list of $n$ real numbers.
  \item A vector in $\RR^{m\times n}$ is a table of real numbers with
    $n$ rows and $m$ columns.  Such a table of real numbers is called
    an $m$ by $n$ \emph{matrix}.  Sometimes $\RR^{m\times n}$ is also
    written as $M_{m,n}$.
  \item A vector in $P_n[x]$ is a polynomial of degree less than or equal
    to $n$ with variable $x$.
  \item A vector in $\RR[[x]]$ is a formal power series (polynomial with
    infinite degree).
  \end{enumerate}
\end{definition}

\begin{problem}
  Do the following:
  \begin{enumerate}
  \item Give two examples of vectors in $\RR^3$ and two examples
    of vectors in $\RR^5$.
  \item Give two examples of vectors in $\RR^{2\times 3}$ and two
    examples of vectors in $\RR^{3\times 3}$.
  \item Give two examples of vectors in $P_2[x]$ and two examples
    of vectors in $P_3[x]$.  Is there a vector in $P_2[x]$ that is
    also a vector in $P_3[x]$?
  \item Give two examples of vectors in $\RR[[x]]$.
  \end{enumerate}
\end{problem}

\begin{definition}[Operations]
  In each of $\RR^n$, $\RR^{m\times n}$, $P_n[x]$, and $\RR[[x]]$, we
  can add two vectors by adding each part separately and we can multiply a
  vector by a scalar (a real number) by taking the number times each component.
\end{definition}

\begin{example}
  We show each operation.
  \begin{enumerate}
  \item For vectors in $\RR^2$, $(2,3)+(4,-2)=(6,1)$ and $3(2,-1)=(6,-1)$.
  \item For vectors in $\RR^{2\times 3}$, 
    \begin{align*}
      \mat{2&3&5\\1&-2&0}+\mat{4&-1&-3\\2&4&6}=\mat{6&2&2\\3&2&6}\text{,
        and}\\
      -2\mat{2&3&5\\1&-2&0}=\mat{-4&-6&-10\\-2&4&0}.
      \end{align*}
    \item For vectors in $P_3[x]$,
      $(1+x+2x^3)+(2x-x^2-x^3)=(1+3x+x^2)$ and
      $3(1+x-2x^3)=(3+3x-6x^3)$.
    \item For vectors in $\RR[[x]]$,
      \begin{align*}
        (1+x+x^2+x^3+\cdots) + (1-x+x^2-x^3+\cdots) =
        (2+2x^2+2x^4+\cdots),\\
        3(x+2x^2+3x^3+\cdots)=3x+6x^2+9x^3+\cdots
      \end{align*}
  \end{enumerate}
\end{example}

\begin{problemtodo}
  Examine the geometry of $c\vec v$, $-\vec u$, $\vec u + \vec v$, and $\vec
  u-\vec v$ in $\RR^2$ and $\RR^3$.
\end{problemtodo}


\begin{definition}
  Let $c_1,c_2,\ldots,c_n$ be scalars (in other words,
  $c_1,c_2,\ldots,c_n$ are real numbers).  A \definemargin{linear
    combination} of vectors $\vec v_1,\, \vec v_2,\, \ldots,\, \vec
  v_n$ is a weighted sum $c_1\vec v_1+c_2\vec v_2+\cdots+c_n\vec v_n$.
\end{definition}

\begin{problemtodo}
  Do some linear combinations in each of the vector spaces
\end{problemtodo}

\begin{problemtodo}\label{prob:geometric spans in Rn}
  Examine the geometry of $\vspan(\vec u)$, $\vspan(\vec u, \vec v)$,
  and $\vspan(\vec u, \vec v, \vec w)$ in $\RR^2$ and
  $\RR^3$ (except don't call it the span, just introduce it as all
  possible linear combinations).  Make sure to look at the span of
  $\vec 0$.
\end{problemtodo}

\begin{problemtodo}
  Represent a real problem as a vector in $\RR^4$ (like quantities of
  different merchandise).  Give a situation where you have a linear
  combination (like production from two suppliers).  Emphasize that
  having a vector with 4 elements is not a problem.
\end{problemtodo}

%%% Local Variables: 
%%% mode: latex
%%% TeX-master: "linear-algebra"
%%% End: 
