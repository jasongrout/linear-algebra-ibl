\documentclass[11pt]{article}
\usepackage[margin=1in]{geometry}

\usepackage{amsmath}
\usepackage{amssymb}
\usepackage{amsthm}
\usepackage{multicol}
\begin{document}
\newcommand{\vecspan}[1]{\mathrm{span}(#1)}
\begin{center}{\centerline\Large Math 80}

{\centering\large 6 Feb 2013, Jason Grout}
\end{center}

The \emph{span} of a set of vectors is the set of all linear combinations of the set of vectors.  For example, $\vecspan{\vec v_1,\vec v_2,\vec v_3}=\{c_1\vec v_1+c_2\vec v_2+c_3\vec v_3 \mid c_1,c_2,c_3\in\mathbb{R}\}$.

\begin{enumerate}
\item For each of the sets of vectors below, give 3 different vectors (if possible) in the set.  Also, draw and describe what the entire set looks like geometrically (e.g., a line, a plane, etc.).
\begin{enumerate}
\begin{multicols}{2}
\item $\vecspan{(1,2)}$ in $\mathbb{R}^2$
\item $\vecspan{(1,2),(1,-1)}$ in $\mathbb{R}^2$
\item $\vecspan{(1,2),(2,4)}$ in $\mathbb{R}^2$
\item $\vecspan{(1,2),(1,-1), (2,1)}$ in $\mathbb{R}^2$


\item $\vecspan{(1,0,-1)}$ in $\mathbb{R}^3$
\item $\vecspan{(1,0,-1), (1,0,1)}$ in $\mathbb{R}^3$
\item $\vecspan{(1,0,-1), (1,0,1), (0,0,2)}$ in $\mathbb{R}^3$
\item $\vecspan{(1,0,-1), (1,0,1), (1,1,0)}$ in $\mathbb{R}^3$

\item $\vecspan{(3)}$ in $\mathbb{R}^1$
\item $\vecspan{(0,0)}$ in $\mathbb{R}^2$
\item $\vecspan{(0,0,0)}$ in $\mathbb{R}^3$
\end{multicols}
\end{enumerate}

\item We saw above that understanding the span depended on knowing if a vector was in the span of some other vectors.  Here we'll tackle that problem.  Write a system of linear equations, one for each component of the vectors, that must be true if $(2,1,0)\in\vecspan{(3,4,2), (1,2,0)}$.  
Also write the corresponding augmented matrix and solve the system (use a calculator or ask me for the RREF).
\item Write a system of equations that must be satisfied if $(-37, -6, -12)\in\vecspan{(-8,3,-3), (7,4,2)}$.
Also write the corresponding augmented matrix and solve the system (use a calculator or ask me for the RREF).
\item Interpret the following system of equations in terms of finding a vector in the span of other vectors.  Solve the system (use a calculator or ask me for the RREF) and interpret your answer.

  \begin{enumerate}
  \item   \begin{align*}
    x-y+z&=5\\
     y-z&=2\\
    3x+2y-2z&=-3
  \end{align*}
\item   \begin{align*}
    5x-y+18z&=53\\
-4x-3y-9z&=-24\\
2x-y+8z&=24
  \end{align*}


\item 
  \begin{align*}
x+2y-3z&=-2\\
-2x-3y+5z&=2\\
5x+14y-19z&=-18\\
  \end{align*}
\end{enumerate}
\end{enumerate}

\end{document}
