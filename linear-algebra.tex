\documentclass[letterpaper,oneside]{book}%

\newif\ifinstructor
%% uncomment one of the following, depending if you want to include the
%% instructor notes or not
\instructortrue
%\instructorfalse


\newif\ifnotes
%% uncomment one of the following, depending if you want to include the
%% instructor notes or not
\notestrue
%\notesfalse



\usepackage[left=1in,right=2.75in,top=1in,bottom=1in]{geometry}
\marginparwidth 1.75in

\usepackage{tabls}
\usepackage{booktabs}
\usepackage{amsmath}
\usepackage{amssymb}
\usepackage{amsthm}
\usepackage{amsfonts}
\usepackage{multicol}
\usepackage{enumitem}
\usepackage{microtype}
\usepackage{tikz}
\usetikzlibrary{positioning}

\usepackage{comment}

\usepackage{graphicx}
\usepackage{wrapfig}

\newcommand{\ds}{\displaystyle}

\let\oldmarginpar\marginpar
\renewcommand\marginpar[1]{\-\oldmarginpar{\raggedright\footnotesize #1}}
%\renewcommand\marginpar[1]{\-\oldmarginpar[\raggedleft\footnotesize #1]{\raggedright\footnotesize #1}}


%\usepackage[12hr]{datetime}
%\newdateformat{draftdate}{%
%\shortdayofweekname{\THEDAY}{\THEMONTH}{\THEYEAR}, %
%\THEDAY\ \shortmonthname[\THEMONTH] \THEYEAR}
%\draftdate
%\usepackage{eso-pic}
%\AddToShipoutPicture{\put(10,10){\small Draft: \today\ at \currenttime }}%--- version: \MakeUppercase{\svnInfoRevision}}}

% Instructor-specific material (answers, helps, etc.)
\ifinstructor
  \newcommand{\instructor}[1]{\marginpar{\textbf{Instructor: }#1}}
\else
  \newcommand{\instructor}[1]{}
\fi

\ifnotes
\renewcommand{\thefootnote}{\roman{footnote}}
\newcommand{\note}[1]{\footnote{#1}\marginpar{\fbox{\textbf{\thefootnote}}}}
\else
\newcommand{\note}[1]{}
\fi


\newcommand{\define}[2][]{\textbf{#2}\index{#1#2}}
\newcommand{\definemargin}[2][]{\marginpar{#2}\define[#1]{#2}}

\theoremstyle{plain}
\newtheorem{theorem}{Theorem}[chapter]
\newtheorem*{theorem*}{Theorem}
\newtheorem{lemma}[theorem]{Lemma}
\newtheorem*{lemma*}{Lemma}
\newtheorem{proposition}[theorem]{Proposition}
\newtheorem{corollary}[theorem]{Corollary}


\newtheoremstyle{box}%
{}{}% standard spacing before and after
{}% Body style
{}{\bfseries}{.}% Heading indent, font, and punctuation
{ }% space after heading
{\thmname{#1}\thmnumber{ #2}\thmnote{: #3}}% head spec

\newtheoremstyle{problem}%
{}{}% standard spacing before and after
{}% Body style
{}{\bfseries}{}% Heading indent, font, and punctuation
{1em}% space after heading
{\fbox{\thmname{#1}\thmnumber{ #2}\thmnote{: #3}}}% head spec

\theoremstyle{box}
\newtheorem{definition}[theorem]{Definition}
\newtheorem{dfn}[theorem]{Definition}
\newtheorem*{definition*}{Definition}
\newtheorem{observation}[theorem]{Observation}
\newtheorem{remark}[theorem]{Remark}
\newtheorem{example}[theorem]{Example}
\newtheorem{question}[theorem]{Question}
\newtheorem*{prep-problems}{Preparation Problems}

%\newtheorem{problem}[theorem]{Problem}
\theoremstyle{problem}
\newtheorem{problemnum}{Task}
\newenvironment{problem}[1][]{\begin{problemnum}[#1]}{\end{problemnum}\nopagebreak\hrule\bigskip}
\newenvironment{problemtodo}[1][]{\begin{problemnum}[#1]\marginpar{\large
      \textbf{TODO}}}{\end{problemnum}\nopagebreak\hrule\bigskip}


% Abbreviations
\newcommand{\ii}{\ensuremath{\vec \imath}}
\newcommand{\jj}{\ensuremath{\vec \jmath}}
\newcommand{\kk}{\ensuremath{\vec k}}
\newcommand{\vv}{\ensuremath{\mathbf{v}}}
\newcommand{\colvec}[1]{\ensuremath{\begin{bmatrix}#1\end{bmatrix}}}
\DeclareMathOperator{\rank}{rank}
\DeclareMathOperator{\rref}{rref}
\DeclareMathOperator{\vspan}{span}
\DeclareMathOperator{\trace}{tr}
\DeclareMathOperator{\proj}{proj}
\DeclareMathOperator{\curl}{curl}
\newcommand{\RR}{\ensuremath{\mathbb{R}}}
% \vp is "vector prime" and corrects spacing when doing something like
% $\vec r'$ (which has the vector and prime almost touching).
% Instead, do something like $\vec r\vp$
\newcommand{\vp}{\ensuremath{^{\,\prime}}}


%The purpose of this code is to allow me to put lines in matrices so that I can create augmented matrices.
\makeatletter
\renewcommand*\env@matrix[1][*\c@MaxMatrixCols c]{%
  \hskip -\arraycolsep
  \let\@ifnextchar\new@ifnextchar
  \array{#1}}
\makeatother

\newcommand{\cl}[1]{  \begin{matrix}  #1  \end{matrix}  }
\newcommand{\mat}[1]{  \begin{bmatrix}  #1  \end{bmatrix}  }
\newcommand{\inv}{^{-1}}
\newcommand{\blank}[1]{\raisebox{0pt}[14pt]{\rule{#1}{1pt}}}


\usepackage{color}
\usepackage[breaklinks]{hyperref}
\hypersetup{
  pdfborder=2,
}

\begin{document}
\frontmatter
\title{Introduction to Linear Algebra}
\author{Jason Grout\thanks{Mathematics Faculty at Drake University,
    \url{jason.grout@drake.edu}} \and Theron
    Hitchman\thanks{Mathematics Faculty at University of Northern
      Iowa, \url{theron.hitchman@uni.edu}}}
\date{Typeset on \today\\
\vfill
\includegraphics[height=1.3cm]{by-sa}
\vfill}
\maketitle
\thispagestyle{empty}

\noindent\copyright{ 2012 Jason Grout and Theron Hitchman.  Some Rights Reserved.\\

\bigskip

\noindent Except where otherwise noted, this work is licensed under the Creative Commons
Attribution-ShareAlike 3.0 United States License. To view a copy of
this license, visit 
\begin{center}
  \url{http://creativecommons.org/licenses/by-sa/3.0/us/}
\end{center}
or send a letter to Creative Commons, 171 Second Street, Suite 300,
San Francisco, California, 94105, USA.

\bigskip

\noindent Please attribute this work to:
\smallskip

 Jason Grout, Mathematics Faculty at Drake University, Des Moines, IA,
 \url{jason.grout@drake.edu}, and

\smallskip

 Theron Hitchman, Mathematics Faculty at University of Northern Iowa,
 Cedar Falls, IA, \url{theron.hitchman@uni.edu}.

\vfill 

The source for this work is available at \url{http://github.com/jasongrout/linear-algebra-ibl}.
\vfill
}
\tableofcontents

\mainmatter

\chapter*{Preface}

Here is an interesting high-level view of the subject: a three-way
interaction between:
\begin{enumerate}
\item $\RR^n$, where our intuition lies
\item $\RR^{m\times n}$, where our computations take place
\item and abstract vector spaces, where we frame our problem and
  interpret results
\end{enumerate}


\chapter{Linear Combinations}
Do each of these in $\RR^n$, $P_n$, $\RR^{m\times n}$, and formal power
  series rings.  Maybe also for $4\times 4$ pixel images (represented
  as matrices)?
\begin{enumerate}
\item Define linear combination.  See geometry in $\RR^n$
\item Matrix-vector multiplication as a shorthand way of writing
  linear combinations in $\RR^n$.
\item Determinant---area/volume (but maybe not needed?  Maybe this
  should be introduced later as a multilinear map?)
\item Calculate inner product, length, angle, orthogonality, and
  projection in each space (except formal power series).  Examine
  these geometrically in $\RR^n$.  This introduces trace and transpose
  for the matrix space.
\end{enumerate}

\section{Vectors in $\RR^n$}

\begin{problem}
  First Problem.
\end{problem}


%%% Local Variables: 
%%% mode: latex
%%% TeX-master: "linear-algebra-ibl"
%%% End: 


\chapter{Vector Spaces}
This unit covers the following ideas. In preparation for the quiz and
exam, make sure you have a lesson plan containing examples that
explain and illustrate the following concepts.  
\begin{enumerate}
\item Objective 1
\item Objective 2
\end{enumerate}
You'll have a chance to teach your examples to your peers prior to the exam.


Again, for $\RR^n$, $P_n$, $\RR^{m\times n}$, and formal power
series. Start emphasizing coordinate vectors more.
\begin{enumerate}
\item What is the span of these vectors?  Emphasize a vector space is
  a span of vectors.

\item Is something in the span?
  \begin{enumerate}
  \item solving systems of equations
  \item Gaussian elimination, LU decomposition
  \item A solution is a linear combination of a vector in the null
    space and a particular solution
  \item polynomial interpolation/Least Squares?
  \end{enumerate}

\item Is this set a span?
  \begin{enumerate}
  \item sets of solutions
  \item eigenvectors
  \item functions that vanish at a certain point
  \end{enumerate}

\item linear independence/dependence

\item basis, coordinate vectors
  \begin{enumerate}
  \item Uniqueness of coordinate vectors
  \item bases have same size
  \item building bases from linearly independent sets or spanning sets
  \item When you have orthonormal bases, finding coordinates becomes
    much easier
  \end{enumerate}

\item changing definition of addition or multiplication (semi-log
  space example)
\end{enumerate}

\begin{problem}
  First Problem.
\end{problem}


%%% Local Variables: 
%%% mode: latex
%%% TeX-master: "linear-algebra"
%%% End: 


\chapter{Linear Transformations}

We want mappings that preserve linear combinations.

\begin{itemize}
\item Define linear transformation
\item Is this a linear transformation?
\item Pick bases, representation with matrices
\item kernel, image (nullspace, columnspace), (one-to-one; onto),
  rank, nullity
\item inverse transformations/matrices
\item change of basis matrix
\item eigenvalues/eigenvectors---nice bases for linear transformations
\item composition of transformations/multiplication of matrices
\item Symmetric matrices and spectral theorem
\end{itemize}

\begin{problem}
  First Problem.
\end{problem}


%%% Local Variables: 
%%% mode: latex
%%% TeX-master: "linear-algebra"
%%% End: 


\chapter{Projects}

This unit provides some ideas for final projects.

\begin{enumerate}
\item DCT
\item Vector fields/ODE connection (Wronskian?  Solutions to
  differential equations?)
\item Page Rank/Markov Processes
\item Software for linear algebra (Matlab, LINPACK, BLAS, MMX, cache
  stages in a CPU and block matrix arithmetic, GPU linear algebra)
\item Pseudoinverses
\item Jordan form
\item QR decomposition
\end{enumerate}

%%% Local Variables: 
%%% mode: latex
%%% TeX-master: "linear-algebra"
%%% End: 


\end{document}

%%% Local Variables:
%%% mode: latex
%%% TeX-master: t
%%% End:
