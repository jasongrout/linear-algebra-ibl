\chapter{Vector Spaces}
This unit covers the following ideas. In preparation for the quiz and
exam, make sure you have a lesson plan containing examples that
explain and illustrate the following concepts.  
\begin{enumerate}
\item Objective 1
\item Objective 2
\end{enumerate}
You'll have a chance to teach your examples to your peers prior to the exam.


Again, for $\RR^n$, $P_n$, $\RR^{m\times n}$, and formal power
series. Start emphasizing coordinate vectors more.
\begin{enumerate}
\item What is the span of these vectors?  Emphasize a vector space is
  a span of vectors.

\item Is something in the span?
  \begin{enumerate}
  \item solving systems of equations
  \item Gaussian elimination, LU decomposition
  \item A solution is a linear combination of a vector in the null
    space and a particular solution
  \item polynomial interpolation/Least Squares?
  \end{enumerate}

\item Is this set a span?
  \begin{enumerate}
  \item sets of solutions
  \item eigenvectors
  \item functions that vanish at a certain point
  \end{enumerate}

\item linear independence/dependence

\item basis, coordinate vectors
  \begin{enumerate}
  \item Uniqueness of coordinate vectors
  \item bases have same size
  \item building bases from linearly independent sets or spanning sets
  \item When you have orthonormal bases, finding coordinates becomes
    much easier
  \end{enumerate}

\item changing definition of addition or multiplication (semi-log
  space example)
\end{enumerate}

\begin{problem}
  First Problem.
\end{problem}


%%% Local Variables: 
%%% mode: latex
%%% TeX-master: "linear-algebra"
%%% End: 
