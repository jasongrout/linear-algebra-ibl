\chapter{Vector Spaces}
\section{Outline}
Again, for $\RR^n$, $P_n$, $\RR^{m\times n}$, and formal power
series. Start emphasizing coordinate vectors more.
\begin{enumerate}
\item What is the span of these vectors?  Emphasize a vector space is
  a span of vectors.

\item Is something in the span?
  \begin{enumerate}
  \item solving systems of equations
  \item Matrix-vector multiplication as a shorthand way of writing
    linear combinations in $\RR^n$.
  \item Gaussian elimination, LU decomposition
  \item A solution is a linear combination of a vector in the null
    space and a particular solution
  \item polynomial interpolation
  \end{enumerate}

\item Is this set a span?
  \begin{enumerate}
  \item sets of solutions
  \item eigenvectors
  \item functions that vanish at a certain point
  \end{enumerate}

\item linear independence/dependence

\item basis, coordinate vectors
  \begin{enumerate}
  \item Uniqueness of coordinate vectors
  \item bases have same size
  \item building bases from linearly independent sets or spanning sets
  \end{enumerate}

\item changing definition of addition or multiplication (semi-log
  space example)
\end{enumerate}

\section{Span}

\begin{definition}
  The \definemargin{span} of a set of vectors is the set of all
  possible linear combinations of the vectors.  In symbols,
  \begin{equation*}
    \vspan(\vec v_1, \vec v_2, \ldots, \vec v_n) := \{c_1\vec
    v_1+c_2\vec v_2+\ldots c_n\vec v_n \st c_1,c_2,\ldots,c_n\in\RR\}
  \end{equation*}

  A \definemargin{vector space} is a span of vectors.  You can think
  of a vector space as a set of vectors in which we can always take a
  linear combination (it is \emph{closed} under linear combinations).

  A set of vectors \define{spans} a set if the set is the span of the
  vectors.
\end{definition}

\begin{problemtodo}
  We already looked at the geometry of spans of one, two, and three
  vectors in \Prob~\ref{prob:geometric spans in Rn}.  Look at a few
  more spans in the other spaces.
\end{problemtodo}

\begin{problemtodo}
  Examine the span of some linearly dependent vectors, and compare
  them with a corresponding basis.  Emphasize that some of the vectors
  were redundant.
\end{problemtodo}

\begin{problemtodo}
  Examine two spans that turn out to be the same set of vectors.
  Discuss why.
\end{problemtodo}

\subsection{Is this a span?}
\label{sec:this-span}

If a set is a span, then it is closed under linear combinations.

\begin{problemtodo}
  functions that vanish at a certain point.  Diagonal matrices.
  Triangular matrices.  See the webwork problem for more ideas.  See
  also the other linear algebra for some examples.  Either show in
  general (and come up with a spanning set), or exhibit a specific
  linear combination counterexample.
\end{problemtodo}

\subsection{Testing if something is in a span}

\begin{problemtodo}
  Test to see if something is in the span of a simple list of vectors,
  say in $\RR^2$.  What is the linear combination that gives the
  target vector?  Write a set of equations that you can solve to give
  you the answer.  Make the answer sufficiently complicated so that
  they can't just get it (like a fraction of one and a fraction of the
  other).
\end{problemtodo}

\begin{definition}
  If a vector $\vec u$ can be written as a linear combination of
  a list of vectors $B=[v_1,v_2,\ldots,v_n]$ as $c_1\vec v_1+c_2\vec v_2+\cdots+c_n\vec
  v_n=\vec u$, then $[\vec u]_B=(c_1,c_2,\ldots,c_n)_B$ is the \emph{coordinate
  vector} of $\vec u$ relative to $B$ (or ``with respect to
$B$'').\note{in this definition, I don't insist that $B$ is a basis
  for the set. This is a bit of a departure from normal terminology,
  but emphasizes the idea of coordinates earlier so that we can get
  comfortable with the term early on.}
\end{definition}

\begin{problemtodo}
  Write the coordinate vector from the problem above.  Explain what it
  means graphically.  Explain what the solution is graphically
  (intersection of the two lines)
\end{problemtodo}

\begin{problemtodo}
  Do another test like above, but this time involving much harder
  vectors, so you can't really just look at the solution and guess it.
  Write the coordinate vector too.  [Hint: set up a system of
  equations and solve.]  Explain what it means graphically in 3d.
\end{problemtodo}

Explain four ways of writing system of equations: equations, vector
equations, matrix equations, and augmented matrices.

\begin{problemtodo}
  Write the system in the previous problem in these 4 ways.
\end{problemtodo}

\begin{problemtodo}
  Make one of the vectors the zero vector.  Ask for two coordinate vectors
\end{problemtodo}

\begin{problemtodo}
  Make the vectors linearly dependent (but not zero).  Ask for two coordinate vectors.
\end{problemtodo}

\begin{problemtodo}
  Ask for a solution where there is no solution.
\end{problemtodo}

Emphasize that there will either be no solution, one solution, or an
infinite number of solutions. (if there are two solutions, then their
average is a solution, and any other point on the line connecting the
two solutions is also a solution).

\begin{problemtodo}
  Is the set of coordinate vectors a subspace for a homogeneous
  system?
\end{problemtodo}

\begin{problemtodo}
  Is the set of coordinate vectors a subspace for a nonhomogeneous system?
\end{problemtodo}

\subsection{Solving systems of linear equations}
\label{sec:solv-syst-line}
The sorts of systems we encountered in the last few \prob{}s are
systems of linear equations.  The linear equations we will be solving
are primarily to determine of a certain vector is in the span of some
other vectors, and if so, what the possible coefficients are.

\begin{definition}
  A system of linear equations is...  A linear equation is...
\end{definition}

\begin{problemtodo}
  Ask to solve two equivalent systems; one the rref of the other.
\end{problemtodo}

There are a series of operations we can do to simplify systems of
equations, but still have the same solution.

\begin{problemtodo}
  Graphically represent each of three operations.
\end{problemtodo}

Gaussian elimination algorithm, and practice it a few times.
Concentrate on following the specific sequence of steps and explain
why that helps you not undo work (zeroing) that you had previously done.

\begin{problemtodo}
  Solve a homogeneous system
\end{problemtodo}

\begin{problemtodo}
  Solve a corresponding nonhomogeneous system.  Point out that the
  solution is always a vector from the homogeneous solution plus a
  particular solution.
\end{problemtodo}

\begin{problemtodo}
  Polynomial interpolation application
\end{problemtodo}

\begin{problemtodo}
  LU Decomposition application?
\end{problemtodo}

\section{Linear Independence}

\begin{problemtodo}
  Repeat a problem where there are multiple solutions to solve for the
  zero vector.  Can we write one of the vectors as a linear
  combination of the others?
\end{problemtodo}

\begin{problemtodo}
  Do a problem where there is just one solution to a homogeneous
  equation.  Can we write one of the vectors as a linear combination
  of the others?  Are there multiple answers?
\end{problemtodo}

\begin{definition}
  A set of vectors is \definemargin{linearly independent} if ...
\end{definition}

\section{Bases}

\begin{problemtodo}
  Size of spanning set can't be too small.  Show that any spanning set
  of $\RR^3$ has at least three vectors, maybe.
\end{problemtodo}

\begin{problemtodo}
  Size of linearly independent set can't be too big.  Show that any
  linearly independent set in $\RR^3$ has at most 3 vectors.
\end{problemtodo}

\begin{definition}
  A \definemargin{basis} is a set that is both linearly independent
  and spans a set.  A basis spans the space, but doesn't have
  redundant vectors.  It is a ``smallest'' spanning set.
\end{definition}

\begin{problemtodo}
  All bases have the same size.
\end{problemtodo}

\begin{problemtodo}
  Find a basis for $\RR^2$, $\RR^3$, and some other spaces (maybe
  relying on previous problems?).  Write some coordinate vectors
  relative to these bases.
\end{problemtodo}

\begin{problemtodo}
  Uniqueness of coordinate vectors when you have a basis.
\end{problemtodo}

\begin{problemtodo}
  Building a basis from a spanning set
\end{problemtodo}

\begin{problemtodo}
  Building a basis from a linearly independent set
\end{problemtodo}

\section{Generalizing Operations}

List the properties (or maybe we should have them verify the
properties for a few spaces).

\begin{problemtodo}
  What if one of the properties wasn't true?
\end{problemtodo}

\begin{problemtodo}
  Examine the semi-log space.  What is the zero vector and additive inverses?
\end{problemtodo}

\begin{problemtodo}
  In the semilog space, was $0\vec v=\vec 0$?  Show this is true in
  general from the axioms.  Was $(-1)\vec v=-\vec v$?  Show it is true
  in general.  The idea of this problem is to help them see that the
  axioms are important for the operations to be intuitive and nice.
\end{problemtodo}

%%% Local Variables: 
%%% mode: latex
%%% TeX-master: "linear-algebra"
%%% End: 
