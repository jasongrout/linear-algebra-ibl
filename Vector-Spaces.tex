\chapter{Vector Spaces}
\section{Outline}
Again, for $\RR^n$, $P_n$, $\RR^{m\times n}$, and formal power
series. Start emphasizing coordinate vectors more.
\begin{enumerate}
\item What is the span of these vectors?  Emphasize a vector space is
  a span of vectors.

\item Is something in the span?
  \begin{enumerate}
  \item solving systems of equations
  \item Matrix-vector multiplication as a shorthand way of writing
    linear combinations in $\RR^n$.
  \item Gaussian elimination, LU decomposition
  \item A solution is a linear combination of a vector in the null
    space and a particular solution
  \item polynomial interpolation
  \end{enumerate}

\item Is this set a span?
  \begin{enumerate}
  \item sets of solutions
  \item eigenvectors
  \item functions that vanish at a certain point
  \end{enumerate}

\item linear independence/dependence

\item basis, coordinate vectors
  \begin{enumerate}
  \item Uniqueness of coordinate vectors
  \item bases have same size
  \item building bases from linearly independent sets or spanning sets
  \end{enumerate}

\item changing definition of addition or multiplication (semi-log
  space example)
\end{enumerate}

\section{Span}

\section{Linear Independence}

\section{Bases}

\section{Generalizing Operations}

%%% Local Variables: 
%%% mode: latex
%%% TeX-master: "linear-algebra"
%%% End: 
